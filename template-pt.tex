\documentclass[letterpaper,12pt,times]{article}

\usepackage[left=1in,right=1in,top=1in,bottom=1in]{geometry}

\usepackage{float}
\usepackage{afterpage}
\usepackage{graphicx}
\usepackage{enumitem}
\usepackage{threeparttable}
\usepackage{pdflscape}

\usepackage{pdfpages}
\newcommand{\pdfscale}[0]{.7647}

\usepackage{microtype}
\usepackage{minesvisual}
\usepackage{multirow}

\usepackage[colorlinks,bookmarksopen,bookmarksnumbered,
citecolor=BlasterBlue,
urlcolor=BlasterBlue,
linkcolor=BlasterBlue,
pdfusetitle
]{hyperref}
\renewcommand{\subsectionautorefname}{Subsection}

\usepackage{fancyhdr}
\usepackage{lastpage}
\usepackage{caption}
\captionsetup[figure]{labelfont={bf},labelformat={default}}
\captionsetup[table]{font={it},labelfont={up,bf},labelformat={default}}
\usepackage{subcaption}
\usepackage{sectsty}
\sectionfont{\centering}

\newcommand{\thead}[1]{\textbf{#1}}
\newcommand{\multithead}[2]{%
  \multicolumn{#1}{|c|}{\thead{#2}}%
}

\usepackage{bibentry}
\nobibliography*

\newcommand{\sectionname}[0]{}
\renewcommand{\sectionmark}[1]{\renewcommand{\sectionname}[0]{#1}}
\newcommand{\subsectionname}[0]{}
\renewcommand{\subsectionmark}[1]{\renewcommand{\subsectionname}[0]{#1}}


% 1: label
% 2: subsection name
% 3: file
\newcommand{\includepdfsubsec}[3]{%
  \clearpage
  \rhead{\thesection{} \sectionname, #2}

  \includepdf[pages={1},scale=\pdfscale,frame=true,pagecommand={\subsection{#2}\label{#1}}]{#3}
  \includepdf[pages={2-},scale=\pdfscale,frame=true,pagecommand={\pagestyle{fancy}}]{#3}
}

\renewcommand{\headrulewidth}{0pt}
\cfoot{{\thepage} {of} {\pageref*{LastPage}}}
\rhead{}
\lhead{}
\rfoot{}
\lfoot{}
\pagestyle{fancyplain}

\renewcommand\thesubsubsection{\thesubsection\roman{subsubsection}}
\renewcommand\thesubsection{\thesection\alph{subsection}.}
\renewcommand\thesection{\arabic{section}.}

\newif\ifinstr
\instrtrue
\usepackage{framed}
\newcommand{\instr}[1]{
  \ifinstr
    \begin{leftbar}%
      \begin{color}{ColoradoRed}%
        \itshape
        \noindent%
        \ignorespaces%
        #1
      \end{color}
    \end{leftbar}%
  \else
  \fi
}

\title{Promotion and Tenure Package}
\author{Hubert J. Farnsworth}
\date{July 1, 3022}

% Uncomment to hide instructions
% \instrfalse

% Uncomment for draft versions. (Please ask senior faculty for
% feedback on your draft package!)
%
% \rfoot{DRAFT: \today}
% \lfoot{DRAFT: \today}

\begin{document}
\maketitle

\thispagestyle{fancyplain}


\renewcommand\contentsname{1. Table of Contents}
\tableofcontents

\listoftables

\listoffigures

\addtocounter{section}{1}


\clearpage
\rhead{\thesection{} \sectionname}
\section{Department Head Memorandum}
\instr{
  (This section is completed by Department Head)

  Memorandum from the Department Head should:

  \begin{enumerate}
  \item name the faculty member for whom the material is being submitted,

  \item state whether the recommendation is for tenure or promotion or
    both, and

  \item certify that all required dossier sections are complete and in
    the order as stated within the Procedures Manual.
  \end{enumerate}
}


\clearpage
\section{Current Faculty Contract}
\instr{
  Include here a copy of the candidate’s most recent, signed Faculty
  Contract. If the candidate or department does not have a copy of the
  Faculty Contract, contact the Office of Academic Affairs.
}

\rhead{\thesection{} Current Faculty Contract}

% \includepdf[pages={1},scale=\pdfscale,frame=true,pagecommand={\section{Current Faculty Contract}}]{files/contract.pdf}
% \includepdf[pages={2-},scale=\pdfscale,frame=true,pagecommand={}]{files/contract.pdf}

\clearpage
\section{Candidate Statement}
\rhead{\thesubsection{} \sectionname, \subsectionname}

\instr{
  Provide a summary of important accomplishments (impacts and
  advances) related to scholarship, teaching, and service in such a
  way as to demonstrate a positive trajectory. Build upon facts,
  describe the unique strengths and circumstances of the
  application. Applications for promotion to Full Professor should
  provide specific examples of leadership and national/international
  recognition.  A narrative length of between two and five pages is
  suggested.
}


% Precite our publications so we can reference publications in the
% candidate statement without messing up the numbering.

\newsavebox\precite
\savebox\precite{\parbox{\textwidth}{\subsection*{Book Chapters}
\begin{itemize}
  \item[\cite{farnsworth3003_paradoxes}] \bibentry{farnsworth3003_paradoxes}.
\end{itemize}
\subsection*{Refereed Journal Papers}
\begin{itemize}
  \item[\cite{farnsworth2999_darkmatter}] \bibentry{farnsworth2999_darkmatter}.
  \item[\cite{farnsworth2997_nanobots}]
    \bibentry{farnsworth2997_nanobots}.
\end{itemize}
\subsection*{Refereed Conference Papers}
\begin{itemize}
  \item[\cite{farnsworth3001_smelloscope}] \bibentry{farnsworth3001_smelloscope}.

\end{itemize}
\subsection*{Workshop Papers (Lightly Refereed)}
\begin{itemize}
  \item[\cite{farnsworth2996_goodnews}] \bibentry{farnsworth2996_goodnews}.

\end{itemize}
\subsection*{Technical Reports}
\begin{itemize}
  \item[\cite{farnsworth3000_doomsday}] \bibentry{farnsworth3000_doomsday}.
\end{itemize}

%%% Local Variables:
%%% mode: latex
%%% TeX-master: "template-pt"
%%% End:
}}

\subsection{Scholarship}

% Summarize of publications for department committee letter
\paragraph{Publications}
I have published or had accepted in total XXX book chapter (at Mines),
XXX refereed journal articles (XXX at Mines), and XXX refereed
conference papers (XXX at Mines).  Of these, my Mines advisees were
lead (or co-lead) authors on XXX refereed journal article and XXX
refereed conference papers, and I was sole author while at Mines XXX.
Additionally, XXX is/are currently under review.


% Summarize of funding for department committee letter
\paragraph{Funding}
At Mines, I have submitted XXX proposals (XXX as lead PI).
%
Of those, XXX were funded (XXX as lead PI), for a total award of
XXX and faculty share of XXX.
%
My research is sponsored by the ABC (XXX awards, XXX total, faculty
share XXX), DEF (XXX awards, XXX total, faculty share XXX),
and GHI (XXX award, XXX total, faculty share XXX).

\paragraph{Research}

\paragraph{Recognition}

\paragraph{Collaboration}

\subsection{Teaching}

\subsection{Service}

\paragraph{Department}

\paragraph{Institution}

\paragraph{External}

\clearpage
\section{Expanded Curriculum Vit\ae}
\subsection{Personal Information}
\instr{
  Should include a brief listing of the following elements:
  \begin{itemize}
  \item Personal Data
    \begin{itemize}
    \item Name
    \item Address (Home)
    \item Citizenship (if applicable)
    \end{itemize}

  \item Education (Name and location of schools; degrees and dates)

  \item Employment History (Date, title and major responsibilities)
  \end{itemize}
}


\subsection{Teaching and Related Activities}
\instr{
  This section should include at a minimum the following components:
  \begin{itemize}
  \item Courses taught (course number and title – no dates)
  \item Course development activities
  \item Teaching research activities
  \end{itemize}

  Sample table summarizing instructional activities (i.e., courses
  taught) is provided on the next page.

  For Research and Library faculty for whom the section is not
  relevant, include on this page the following sentence; “Not relevant
  for promotion consideration of faculty member.”
}


\begin{table}[b]
\caption{Courses Taught at Mines}
\label{tab:courses}
\centering
\begin{tabular}{|cp{5.45cm}cp{1cm}p{4.75cm}|}
\hline
\thead{Number}
& \multicolumn{1}{c}{\thead{Title}}
& \thead{Type}
& \thead{Credit Hours}
& \multicolumn{1}{c|}{\thead{Notes}}\\
\hline
\hline
  XXX YYY & ZZZ & Lecture & \multicolumn{1}{c}{3}
& Required core course for CS undergraduate students
  \\
  \hline
\end{tabular}
\end{table}


\afterpage{
\begin{landscape}
\begin{table}[H]
\centering
\caption{Course Summary for XXXYYY: ZZZ}
\begin{small}
\begin{tabular}{|cccp{2cm}cp{11.5cm}|}
  \hline
\thead{Term}
& \thead{Section}
& \thead{\% Resp.}
& \thead{Num. of St.}
& \thead{``Eff.''}
& \thead{Comments}
\\
\hline
\hline
  S3021
  & XXXYYY
  & 100\%
  & XXX
  & XXX (XXX)
  & XXX
    \\
  \hline

\end{tabular}
\end{small}
\end{table}
\end{landscape}
}

\subsection{Scholarly Activities}

\instr{
This section should include at a minimum the following components:
\begin{itemize}
    \item Students advised (M.S., Ph.D.; thesis title and date)
    \item Unfunded research activities (title, dates; students
      involved---name, degree)
    \item Funded research activities (project title, agency, dates,
      support level; students involved – name, degree)
    \item Other research contributions (service on graduate
      committees, interdisciplinary interactions)
    \end{itemize}

Sample table summarizing student advising information is provided below.


Information related to external funding activities should include the following:

\begin{itemize}
\item Clearly state the total funding that the candidate has been
  involved in securing as well as the individual’s total share.

\item Clearly identify the amount of funding that is credited to the
  candidate in each multi-investigator grant.

\item Clearly identify the candidates’ role on each funding award,
  e.g. PI, co-PI, senior investigator, etc.

\item In collaborative grants with outside institutions, identify the
  amount expended at CSM.

\item List non-funded proposals (same level of detail as funded
  proposals) to demonstrate track record for trying to obtain grant
  funding.
\end{itemize}

Sample table summarizing some relevant grant information is provided
on the following page.

For Teaching and Library faculty for whom the section is not
relevant, include on this page the following sentence; “Not relevant
for promotion consideration of faculty member.”
}

\subsubsection*{Student Advising}

\begin{table}[H]
  \caption{Graduate Student Advising}
  \centering
\begin{tabular}{|cccc|}
  %\hline
  %\multithead{4}{Graduate Student Advising} \\
  \hline
  \thead{Student Name}
  & \thead{Faculty Role}
  & \thead{Degree \& Year}
  & \thead{Funding Source}
  \\
  \hline
  \hline
  Blaster T. Burro
  & Advisor
  & Ph.D., Est. 2024
  & Faculty Grant
  \\
  \hline
\end{tabular}
\end{table}


\autoref{tab:grants} lists my funded grants.
%
\autoref{tab:proposal:review} lists proposals under review.
%
\autoref{tab:proposal:unfunded1} lists unfunded proposals.


\begin{table}[H]
  \caption{Undergraduate Student Advising}
  \centering
\begin{tabular}{|lll|}
  % \hline
  % \multithead{3}{Undergraduate Student Advising} \\
  \hline
  \thead{Student Name}
  & \thead{Structure}
  & \thead{Project}
  \\
  \hline
  \hline
  Marvin T. Miner
  & MURF
  & XXX
  \\
  \hline
\end{tabular}
\end{table}

\afterpage{
\begin{landscape}
\begin{table}[H]
  \caption{Funded Grants}
  \label{tab:grants}
  \centering
  \begin{small}
\begin{tabular}{|p{9cm}cccccp{4cm}|}
  % \hline
  % \multithead{7}{Funded Grants}\\
  \hline
 \thead{Title}
  & \thead{Sponsor}
  & \thead{Dur.}
  & \thead{Amount}
  & \thead{PI Share}
  & \thead{Role}
  & \thead{Co-PIs}
    \\
  \hline
  \hline
  XXX: YYY: ZZZ
  & NSF
  & 3 yr
  & XXX
  & XXX
  & Co-PI
  & XXX (PI), YYY
  \\
  \hline
  \hline
  &
  & Total:
  & \$XXX
  & \$XXX
  &
  &
  \\
  \hline
\end{tabular}
\end{small}
\end{table}


\begin{table}[H]
  \caption{Proposals Under Review}
  \label{tab:proposal:review}
  \centering
  \begin{small}
\begin{tabular}{|p{9cm}cccccp{4cm}|}
  \hline
  \thead{Title}
  & \thead{Sponsor}
  & \thead{Dur.}
  & \thead{Amount}
  & \thead{PI Share}
  & \thead{Role}
  & \thead{Co-PIs}
    \\
  \hline
  \hline
  XXX: YYY: ZZZ
  & NSF
  & 3 yr
  & XXX
  & XXX
  & Co-PI
  & XXX (PI), YYY
  \\
  \hline
  \hline
  &
  & Total:
  & \$XXX
  & \$XXX
  &
  &
  \\
  \hline
\end{tabular}
\end{small}
\end{table}

\begin{table}[H]
  \caption{Unfunded Proposals}
  \label{tab:proposal:unfunded1}
  \centering
  \begin{small}
\begin{tabular}{|p{9cm}cccccp{4cm}|}
  \hline
  \thead{Title}
  & \thead{Sponsor}
  & \thead{Dur.}
  & \thead{Amount}
  & \thead{PI Share}
  & \thead{Role}
  & \thead{Co-PIs}
    \\
  \hline
  \hline
  XXX: YYY: ZZZ
  & NSF
  & 3 yr
  & XXX
  & XXX
  & Co-PI
  & XXX (PI), YYY
  \\
  \hline
  \hline
  &
  & Total:
  & \$XXX
  & \$XXX
  &
  &
  \\
  \hline
\end{tabular}
  \end{small}
\end{table}
\end{landscape}
}


\subsubsection*{Thesis Committees}

\begin{table}[H]
  \caption{Thesis Committees}
\begin{tabular}{|lp{3in}ll|}
  \hline
  \thead{Name} & \thead{Thesis Title} & \thead{Degree} & \thead{Year}\\
  \hline
  \hline
  Blaster T. Burro & Deleterious Effects of Sunlight on Multi-Generational
                     Subterranean Burro Populations.
                                      & M.S. & 1918 \\
  \hline
\end{tabular}
\end{table}

\subsection{Publications and Presentations}

\instr{
The following items should be included in this section:

\begin{itemize}
\item Books
\item Refereed Journals
\item Published in conference proceedings
\item Published scientific discussions
\item Published abstracts
\item Book reviews
\item Reports
\item Presentations
\item Other
\end{itemize}

 These should conform to the following format requirements

 \begin{itemize}
 \item Provide separate lists of archival journal publications,
   book/book chapters, and conference proceedings. Clearly identify
   publications that are peer reviewed and those that are not.

 \item Clearly mark all co-authors who are CSM students and CSM
   post-doctorals. In the example below, CSM students by *:\\
   J. R. Smith, A. Gables, P.T. Barnum*, ``Interesting Research
   Advances'', J. Important Research, 1, 40-48 (2013). DOI or
   identifying link
 \item If available, as part of the reference, also provide a
   hyperlink to an electronic version of the publication.
 \end{itemize}


 For Teaching and Library faculty for whom the section is not
 relevant, include on this page the following sentence; ``Not relevant
 for promotion consideration of faculty member.''
}

Advised and Co-advised Mines student authors are identified with a
$\star$ after the name.  Other Mines student authors identified with
$\circ$ after the name.

\subsection*{Book Chapters}
\begin{itemize}
  \item[\cite{farnsworth3003_paradoxes}] \bibentry{farnsworth3003_paradoxes}.
\end{itemize}
\subsection*{Refereed Journal Papers}
\begin{itemize}
  \item[\cite{farnsworth2999_darkmatter}] \bibentry{farnsworth2999_darkmatter}.
  \item[\cite{farnsworth2997_nanobots}]
    \bibentry{farnsworth2997_nanobots}.
\end{itemize}
\subsection*{Refereed Conference Papers}
\begin{itemize}
  \item[\cite{farnsworth3001_smelloscope}] \bibentry{farnsworth3001_smelloscope}.

\end{itemize}
\subsection*{Workshop Papers (Lightly Refereed)}
\begin{itemize}
  \item[\cite{farnsworth2996_goodnews}] \bibentry{farnsworth2996_goodnews}.

\end{itemize}
\subsection*{Technical Reports}
\begin{itemize}
  \item[\cite{farnsworth3000_doomsday}] \bibentry{farnsworth3000_doomsday}.
\end{itemize}

%%% Local Variables:
%%% mode: latex
%%% TeX-master: "template-pt"
%%% End:


\subsection{Honors, Awards, and Recognitions}


\instr{
  Provide information on any honors, awards and recognitions received
  both internal and external to Mines.
}


\subsection{Service and Mentoring Activities}

\instr{
  Provide here a list of service and/or mentoring activities done by
  the faculty member in support of the Department, Mines, and/or
  external professional organizations. This should include:

  \begin{itemize}
    \item National and international committees, editorial boards, panels, review teams, etc.
    \item Departmental and campus committees, graduate student
      committees, junior faculty mentoring, assessment activities,
      accreditation activities, student engagement and retention
      activities, student group advising, activities in partnership
      with Student Life, etc.
    \item Professional societies
    \item Outreach activities
    \item Organizing conferences, sessions, workshops, etc.
  \end{itemize}
}



\clearpage
\section{Letter of Recommendation from Departmental Committee}
\rhead{\thesection{} \sectionname}


\instr{
  (This section is completed by Department Committee)

  (Departmental Committee inserts their letter(s) of recommendation
  here) As directed by the Faculty Handbook, the Departmental
  Promotion (and Tenure) Committee reviews the application package and
  provides a recommendation(s) in writing to the Department Head that
  is included here.

  In preparing this recommendation, the Committee should consider the
  criteria for tenure and/or promotion listed in the appropriate
  section of the Faculty Handbook and is encouraged to address the
  specific items listed in Section 6.5 of Academic Affairs Procedures
  Manual.

  The letter of recommendation(s) must list the names of all members
  of the Departmental Promotion (and Tenure) Committee and be signed
  by all members who participated in making the recommendation.  At
  least $\frac{3}{4}$ of the eligible members of the Committee must participate in
  the decision (participation in the tenure/review process is a
  required service activity for all eligible committee members that
  are not on sabbatical or extended sick leave). The final vote
  (unanimous, or a number for or against the candidate’s request for
  promotion and/or tenure) should be given.

  If so desired, a separate letter prepared by members of the
  Committee holding a minority view point may also be prepared and
  included in this section.
}


\clearpage
\section{Letter of Recommendation from Department Head}
\rhead{\thesection{} \sectionname}
\instr{
  (This section is completed by Department Head)

  (Department Head inserts her/his letter of recommendation here)

  As directed by the Faculty Handbook, the Department Head reviews the
  application package and provides a recommendation(s) in writing to
  the appropriate University Promotion and/or Tenure Committee that is
  included here.

  In preparing this recommendation, the Department Head should
  consider the criteria for tenure and/or promotion listed in the
  appropriate section of the Faculty Handbook and is encouraged to
  address the specific items listed in Section 6.5 of Academic Affairs
  Procedures Manual.
}

\clearpage
\section{Letter of Recommendation from College Dean}
\rhead{\thesection{} \sectionname}

\instr{
  (This section is completed by College Dean)

  (Department Head inserts her/his letter of recommendation here)

  As directed by the Faculty Handbook, the College Dean reviews the
  application package and provides a recommendation(s) in writing to
  the appropriate University Promotion and/or Tenure Committee that is
  included here.

  In preparing this recommendation, the Dean should consider the
  criteria for tenure and/or promotion listed in the appropriate
  section of the Faculty Handbook and is encouraged to address the
  specific items listed in Section 6.5 of Academic Affairs Procedures
  Manual.
}


\clearpage
\section{Performance Evaluations}
\rhead{\thesection{} \sectionname}
%
\instr{
  Insert Faculty Evaluation Summary sheets completed by the Department
  Head for the past three most recent years.
}

% \includepdfsubsec{sec:perf:2019}{2019 (Fillable PDF)}{files/2019-eval-print.pdf}

\clearpage
\section{External Evaluation Letters}
\rhead{\thesection{} \sectionname}
% External evaluation letters omitted for preliminary review.
% \input{sec/external-letter.tex}

\clearpage
\section{Submission Narratives}
\rhead{\thesubsection{} \sectionname, \subsectionname}

\subsection{Teaching Accomplishments}

\instr{
  This section should include information related to the following items:

  \begin{enumerate}
  \item a brief summary of student evaluation of teaching. If available,
    this should include information other than student course
    evaluations (e.g., class visits, course portfolios, etc.). Note the
    expected teaching load in your department for faculty of your rank
    at the undergraduate and/or graduate levels.

  \item a portfolio of teaching materials. This should consist of a brief
    (250-300 word) teaching philosophy statement followed by materials
    that show the application of learning objectives, which may include
    sample lesson plans, syllabi, sample graded student work, and/or
    activities. It is not necessary to include all such materials, but
    the candidate should select those materials that demonstrate
    teaching excellence. Example assessment measures should also be
    included to illustrate ways that the instructor evaluated the
    efficacy of the curriculum, addressing: (a) rationale for the
    changes/development, (b) evidence of how material changed (c)
    assessment results – did the development accomplish the desired
    goals? How do you know?

  \item a brief description of innovative course development activities
    and practices.

  \item data on undergraduate student advising (level of effort,
    co-advisors, outcomes). Differentiate undergraduate research
    advising from conventional advising. For undergraduate research
    advising, list student names, research period, graduation semester,
    project title, and outcomes, such as conference presentations or
    publications.

  \item information related to workshops and short courses, number of
    attendees, and your role (e.g., organizer, lecturer, one of two
    instructors for three day course, etc.).
  \end{enumerate}

  For Research and Library faculty for whom the section is not relevant,
  include on this page the following sentence; ``Not relevant for
  promotion consideration of faculty member.''
}


\subsection{Scholarly Achievements}

\instr{
  Provide here a brief narrative describing evidence of scholarly
  achievement in the your academic field and/or pedagogical
  development. This narrative should, at a minimum:


  \begin{itemize}
  \item provide information regarding the quality of journals in which the
    candidate has published his/her work,

  \item provide acceptance rates and/or impact factors (or any other
    published quality indicators or measures). If acceptance rates are
    difficult to obtain from the Internet, candidates may consider
    contacting journal editors directly,

  \item provide information related to authorship conventions in your
    field or subfield as appropriate (i.e., define how authorship order
    determined for multi-author publications). If authorship order is
    alphabetical, provide additional narrative defining role of
    applicant in publications listed, and


  \item if writing a book, text or otherwise, in your field or subfield is
    considered an important scholarly contribution.
  \end{itemize}

  Please limit the length of this narrative to no more than two pages.

  For Teaching and Library faculty for whom the section is not
  relevant, include on this page the following sentence; ``Not relevant
  for promotion consideration of faculty member.''
}


\subsection{External Funding Raised}

\instr{
Provide here a brief narrative describing evidence of external
funding activities.

Provide information on the competitiveness of the
funding sources.

Provide information on how the funds were
utilized to support students, post-doctoral, and/or
technicians.

Identify products such as number of papers, software, workshops,
book, patents, etc. that resulted from the funding generated.  This
should reference and agree with your publication record as evidence
in you CV in section 11.
}

\subsection{Student Advising}
\instr{
  If applicable, provide a brief narrative of graduate student
  advising activities. Include in this narrative commentary on: 1)
  funding activities used to support students, 2) student completion
  rates, 3) scholarly activities (e.g., publications, presentations,
  etc.) related to student work, and 4) any special items of merit or
  note that students have obtained under your mentorship.

  Please limit the length of this narrative to no more than two pages.

  For Teaching and Library faculty for whom the section is not
  relevant, include on this page the following sentence; ``Not
  relevant for promotion consideration of faculty member.''
}


\subsection{Service and Mentoring Activities}

\instr{
  Provide here a brief summary of service and/or mentoring activities
  done by the faculty member in support of the Department, Mines,
  and/or external professional organizations. For all of the items
  listed in the curriculum vitae the candidate should state his/her
  level of effort and the impact of the contributions.

  Please limit the length of this narrative to no more than two pages.
}


\subsection{Other Information}

\instr{
  Include here any other additional information the candidate believes
  is relevant for consideration in his/her promotion/tenure
  application.

  Please limit the length of this narrative to no more than two pages.
}

\clearpage

\appendix

\renewcommand\thesection{\Alph{section}.}

\rhead{\thesection{} \sectionname}

\section{Impact of COVID}


\clearpage
\section{Sample Publications} %
\label{sec:samplepubs}

\clearpage
\rhead{\thesection{} \sectionname}
\section{Additional Evaluations of Teaching} %
\label{sec:eval}

This section contains additional evaluations of my teaching beyond the
student course evaluations.

\section{Sample Teaching Materials}
\rhead{\thesection{} \sectionname}



\bibliographystyle{IEEEtran}

%% Include the bib, but hide it.
%% TODO: maybe a hyper-ref conflict?
\newsavebox\mytempbib
\savebox\mytempbib{\parbox{\textwidth}{\bibliography{template-pt-ref}}}

\end{document}

%%% Local Variables:
%%% mode: latex
%%% TeX-master: t
%%% End:
